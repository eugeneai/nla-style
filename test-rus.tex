\documentclass[12pt]{llncs}
\usepackage[T2A]{fontenc}
\usepackage[utf8]{inputenc}
\usepackage[english,russian]{babel}
\usepackage[russian]{nla}

%\usepackage[english,russian]{nla}

% \graphicspath{{pics/}} %Set the subfolder with figures (png, pdf).

\usepackage{showframe}
\begin{document}

\title{Название доклада\thanks{Работа выполнена при поддержке РФФИ (РНФ, другие фонды), проект \textnumero~00-00-00000.}}
% Первый автор
\author{И.~О.~Фамилия\inst{1}
  \and
% Второй автор
  И.~О.~Фамилия\inst{2}
  \and
% Третий автор
  И.~О.~Фамилия\inst{2}
} % обязательное поле
\institute{Институт (название в краткой форме), Город, Страна\\
  \email{email}
  \and
Институт (название в краткой форме), Город, Страна\\
\email{email}}
% Другие авторы...

\maketitle

\begin{abstract}
Аннотация на русском языке.

\keywords{ключевое слово1, другие}
\end{abstract}

\section{Основные результаты} % не обязательное поле

Текст доклада на русском языке.

Научный стиль имеет ряд общих черт, проявляющихся независимо от характера определённых наук (естественных, точных, гуманитарных) и различий между жанрами высказывания (монография, статья, доклад, учебник, курсовая работа и т. д.), что даёт возможность говорить о специфике стиля в целом. Вместе с тем вполне естественно, что, например, тексты по физике, химии, математике заметно отличаются по характеру изложения от текстов по филологии или истории.

Научный стиль характеризуется логической последовательностью изложения, упорядоченной системой связи между частями высказывания, стремлением авторов к точности, сжатости, однозначности при сохранении насыщенности содержания.

    Логичность — это, по возможности, наличие смысловых связей между последовательными единицами (блоками) текста.
    Последовательностью обладает только такой текст, в котором выводы вытекают из содержания, они непротиворечивы, текст разбит на отдельные смысловые отрезки, отражающие движение мысли от частного к общему или от общего к частному.
    Ясность, как качество научной речи, предполагает понятность, доступность.

Лексика научного стиля речи

Так как ведущей формой научного мышления является понятие, почти каждая лексическая единица в научном стиле обозначает понятие или абстрактный предмет. Точно и однозначно называют специальные понятия научной сферы общения и раскрывают их содержание особые лексические единицы — термины. Термин — это слово или словосочетание, обозначающее понятие специальной области знания или деятельности и являющееся элементом определенной системы терминов. Внутри данной системы термин стремится к однозначности, не выражает экспрессии. Однако это не означает его стилистической нейтральности. Термину, также как и многим другим лексическим единицам, присуща стилистическая окрашенность (научного стиля), которая отмечается в виде стилистических помет в соответствующих словарях. Приведем примеры терминов: «атрофия», «численные методы алгебры», «диапазон», «зенит», «лазер», «призма», «радиолокация», «симптом», «сфера», «фаза», «низкие температуры», «керметы». Значительная часть терминов является интернациональными словами.

В количественном отношении в текстах научного стиля термины преобладают над другими видами специальной лексики (номенклатурными наименованиями, профессионализмами, профессиональными жаргонизмами и пр.); в среднем терминологическая лексика обычно составляет 15-20 % общей лексики научного стиля. В приведенном фрагменте научно-популярного текста термины выделены особым шрифтом, что позволяет увидеть их количественное преимущество по сравнению с другими лексическими единицами:
« 	К тому времени физики уже знали, что эманация — это радиоактивный химический элемент нулевой группы периодической системы, то есть — инертный газ; порядковый номер его — 86, а массовое число наиболее долгоживущего изотопа — 222. 	»

Для терминов, как основных лексических составляющих научного стиля речи, а также для других слов научного текста, характерно употребление в одном, конкретном, определенном значении. Если слово многозначно, то оно употребляется в научном стиле в одном, реже — в двух значениях, которые являются терминологическими: сила, размер, тело, кислый, движение, твердый (Сила — величина векторная и в каждый момент времени характеризуется числовым значением). Обобщенность, абстрактность изложения в научном стиле на лексическом уровне реализуется в употреблении большого количества лексических единиц с абстрактным значением (абстрактная лексика)[1]. Научный стиль имеет и свою фразеологию, включающую составные термины: «солнечное сплетение», «прямой угол», «наклонная плоскость», «глухие согласные», «деепричастный оборот», «сложносочиненное предложение», а также различного рода клише: «заключается в …», «представляет собой …», «состоит из …», «применяется для …» и пр.
Морфологические особенности научного стиля речи

Языку научного общения присущи свои грамматические особенности. Отвлеченность и обобщенность научной речи проявляются в особенностях функционирования разнообразных грамматических, в частности морфологических, единиц, что обнаруживается в выборе категорий и форм, а также степени их частоты в тексте. Формы единственного числа имен существительных используются в значении множественного числа: «волк — специализированное хищное животное из класса млекопитающих из рода волков из семейства псовых»; «липа начинает цвести в конце июня». Вещественные и отвлеченные существительные нередко употребляются в форме множественного числа: смазочные масла, шумы в радиоприемнике, большие глубины.

Названия понятий в научном стиле преобладают над названиями действий, это приводит к меньшему употреблению глаголов и большему употреблению существительных. При использовании глаголов заметна тенденция к их десемантизации, то есть утрате лексического значения, что отвечает требованию абстрактности, обобщенности научного стиля изложения. Это проявляется в том, что большая часть глаголов в научном стиле функционирует в роли связочных: «быть», «являться», «называться», «считаться», «стать», «становиться», «делаться», «казаться», «заключаться», «составлять», «обладать», «определяться», «представляться» и др. Имеется значительная группа глаголов, выступающих в качестве компонентов глагольно-именных сочетаний, где главная смысловая нагрузка приходится на имя существительное, называющее действие, а глагол выполняет грамматическую роль (обозначая действие в самом широком смысле слова, передает грамматическое значение наклонения, лица и числа): приводить — к возникновению, к гибели, к нарушению, к раскрепощению; производить — расчеты, вычисления, наблюдения. Десемантизация глагола проявляется также в преобладании в научном тексте глаголов широкой, абстрактной семантики: существовать, происходить, иметь, появляться, изменять(ся), продолжать(ся) и пр.

Для научной речи характерно использование глагольных форм с ослабленными лексико-грамматическими значениями времени, лица, числа, что подтверждается синонимией структур предложения: перегонку производят — перегонка производится; можно вывести заключение — выводится заключение и пр.

Еще одна морфологическая особенность научного стиля состоит в использовании настоящего вневременного (с качественным, признаковым значением), что необходимо для характеризации свойств и признаков исследуемых предметов и явлений: «при раздражении определенных мест коры больших полушарий регулярно наступают сокращения»; «углерод составляет самую важную часть растения». В контексте научной речи вневременное значение приобретает и прошедшее время глагола: «Произведено n опытов, в каждом из которых x принял определенное значение». По наблюдениям ученых, процент глаголов настоящего времени в три раза превышает процент форм прошедшего времени, составляя 67-85 \% от всех глагольных форм.

Отвлеченность и обобщенность научной речи проявляется в особенностях употребления категории вида глагола: около 80 \% составляют формы несовершенного вида, являясь более отвлеченно-обобщенными. Немногие глаголы совершенного вида используются в устойчивых оборотах в форме будущего времени, которое синонимично настоящему вневременному: «рассмотрим…», «уравнение примет вид». Многие глаголы несовершенного вида лишены парных глаголов совершенного вида: «Металлы легко режутся».

Формы лица глагола и личные местоимения в научном стиле также употребляются в соответствии с передачей отвлеченно-обобщающих значений. Практически не используются формы 2-го лица и местоимения ты, вы, так как они являются наиболее конкретными, мал процент форм 1-го лица ед. числа. Наиболее часты в научной речи отвлеченные по значению формы 3-го лица и местоимения он, она, оно. Местоимение мы, кроме употребления в значении так называемого авторского мы, вместе с формой глагола часто выражает значение разной степени отвлеченности и обобщенности в значении «мы совокупности» (я и аудитория): Мы приходим к результату. Мы можем заключить.


% Список литературы.
\begin{thebibliography}{99}
\bibitem{1}
% Format for Journal Reference
Author1 N., Author2 N.  {\it Article title}. Journal. Year. Vol.~Volume, No~Number. Pp.~Page numbers.
% Format for books
\bibitem{2}
Author N. Book title. Place: Publisher, year.
% Format for Russian Journal Reference
\bibitem{3} Фамилия И.О. {\it Название статьи}. Журнал. Год. Т.~том,  \textnumero~номер. С.~страницы.
\bibitem{31} Фамилия И.О. {\it Название статьи}. Журнал. Год. Т.~том,  \textnumero~номер. С.~страницы.
\bibitem{32} Фамилия И.О. {\it Название статьи}. Журнал. Год. Т.~том,  \textnumero~номер. С.~страницы.
\bibitem{33} Фамилия И.О. {\it Название статьи}. Журнал. Год. Т.~том,  \textnumero~номер. С.~страницы.
\bibitem{34} Фамилия И.О. {\it Название статьи}. Журнал. Год. Т.~том,  \textnumero~номер. С.~страницы.
\bibitem{35} Фамилия И.О. {\it Название статьи}. Журнал. Год. Т.~том,  \textnumero~номер. С.~страницы.
\bibitem{36} Фамилия И.О. {\it Название статьи}. Журнал. Год. Т.~том,  \textnumero~номер. С.~страницы.
\bibitem{37} Фамилия И.О. {\it Название статьи}. Журнал. Год. Т.~том,  \textnumero~номер. С.~страницы.
\bibitem{38} Фамилия И.О. {\it Название статьи}. Журнал. Год. Т.~том,  \textnumero~номер. С.~страницы.
\bibitem{39} Фамилия И.О. {\it Название статьи}. Журнал. Год. Т.~том,  \textnumero~номер. С.~страницы.
\bibitem{30} Фамилия И.О. {\it Название статьи}. Журнал. Год. Т.~том,  \textnumero~номер. С.~страницы.
\bibitem{3g} Фамилия И.О. {\it Название статьи}. Журнал. Год. Т.~том,  \textnumero~номер. С.~страницы.
\bibitem{3b} Фамилия И.О. {\it Название статьи}. Журнал. Год. Т.~том,  \textnumero~номер. С.~страницы.
\bibitem{3c} Фамилия И.О. {\it Название статьи}. Журнал. Год. Т.~том,  \textnumero~номер. С.~страницы.
% etc
\end{thebibliography}



\begin{englishtitle}
\title{Title of the Paper}
% First author
\author{Name FamilyName1\inst{1},
Name FamilyName2\inst{2}}
\institute{Affiliation, City, Country\\
  \email{email}
  \and
Affiliation, City, Country\\
\email{email}}
% etc

\maketitle

\begin{abstract}
Insert your english abstract here. Include 3-6 keywords below.

\keywords{keyword1,  \emph{etc.}}
\end{abstract}
\end{englishtitle}


\end{document}

%%% Local Variables:
%%% mode: latex
%%% TeX-master: t
%%% End:
