\title{Имитационное логическое моделирование группой лифтов}

\author{
  С.~В.~Арляпов\inst{1}
  \and
  Е.~А.~Черкашин\inst{2,3,4}
}

\institute{Иркутский государственный университет путей и сообщения, Иркутск, Россия
  \and
  Институт динамики систем и теории управления им.~В.~М.~Матросова СО РАН, Иркутск, Россия
  \and Научный исследовательский иркутский государственный технический университет, Иркутск, Россия
  \and
  Институт математики, экономики и информатики ИГУ, Иркутск, Россия\\
  \email{sava\_arl@mail.ru,eugeneai@icc.ru}
}

\maketitle

\begin{abstract}
Рассмотрена задача разработки имитационной модели управления группой лифтов, реализованная при помощи логического языка программирования. Модель состоит двух частей: первая -- физическая модель лифта, вторая - логическая модель принятия решения по управлению лифтом. Произведены тестовые расчеты.

\keywords{автоматизированная система управления, имитационное моделирование, логические языки программирования, Пролог}
\end{abstract}

Рассматривается задача синтеза плана управления системой лифтов высотного здания. Данная задача описана в книге С.В. Васильева [1], где приводится абстрактная модель, логическая модель и логический вывод, демонстрирующий возможность применения логического подхода в синтезе управления. В [2] представлены результаты разработки системы автоматического доказательства теорем (АДТ) на основе теории позитивно-образованных формул. Реализация поставленной в [1] задачи на программном обеспечении из [2] достаточно трудоемко, что привело к решению реализовать задачу на существующими средствами программирования логического.

Целью данного исследования является выявление принципиальной возможности построения и использования системы логического вывода, основанного на методе резолюции и хорновских дизъюнктах, в рассматриваемой задаче. При этом по сравнению с исходной задачей неизбежны некоторые упрощения.

Первое допущение -- дискретность виртуального времени с заданной величиной интервала между соседними моментами времени (тактами). Множество моментов времени счетно и содержит начальный момент. Высота этажей считается одинаковой, и скорость движения лифта с одного этажа на другой полагаем равной одному такту. Длительность остановки кабин для входа-выхода пассажиров равняется одному такту.  Так же не рассматриваются случаи переполнения кабин.

Считается, что поведение пассажира основано на следующих правилах:
\begin{enumerate}
\item для вызова лифта он нажимает на этаже кнопку вызова и ждёт кабину;
\item войдя в кабину, пассажир задаёт ей команду, для чего он нажимает кнопку нужного этажа, который вносится в маршрут данной кабины.
\end{enumerate}

Простейшим алгоритмом принятия решения является поиск ближайшей кабины к месту вызова. %Однако, термин «ближайшая» требует уточнения и рассмотрения примера.
Допустим, есть система из $k = 2$ кабин, способных перемещаться по \(n = 5\) этажам. Пусть кабины находятся на 1-м и 2-м этажах, первая пуста и находится в покое, а второй предстоят остановки на 3-м и 4-м этажах. Поступает вызов с пятого этажа, и первая кабина получается ближайшей, так как её требуется 4 такта, а второй кабине требуется 5 тактов.  Таким образом, дистанция -- это количество тактов, которое необходимо кабине, чтобы добраться до этажа, выполняя уже сформированный маршрут.

Есть и другой подход, который основывается на исключении вариантов развития системы в виртуальном будущем, полученных при помощи логического вывода. И если после сокращения допустимых альтернатив их останется несколько, то выбор может быть случайным или основываться на каких-либо количественных критериях, например, среднем времени ожидания обработки вызова.

Основными объектами в данной модели являются кабина $cab$ и человек $man$. В момент времени $t$ кабина имеет вид $cab(i, e, S, t)$, где $i$ -- идентификатор кабины, $e$ -- этаж, а $S$ -- маршрут кабины, список этажей. Человек задается как $man(e, d, \tau, t)$, где $e$ -- этаж, $d$ -- целевой этаж, который добавляется в маршрут $S$ в момент входа человека в кабину и $d \neq e$, $\tau$ -- длительность ожидания человеком кабины. Дистанцией же будет $dist(e, S, d, i, \alpha)$, где $\alpha$ -- это дистанция от кабины $i$ на этаже $е$ с маршрутом $S$ до этажа $d$, где произошёл вызов.

Модель разбита на две части: систему взаимодействующих объектов и алгоритм принятия решений. Разработку обеих частей можно вести независимо дуг от друга.  Реализация подсистем может осуществляться на разных языках программирования, например: блок имитации на объектно-ориентированном языке Питон, а логический блок на языке Пролог, который поддерживает декларативное программирование. В последствии планируется заменить логический блок на систему автоматического доказательства теорем [2], и, таким образом, приблизить результат к исходной задаче.


\begin{thebibliography}{9}
\bibitem{vas200} С. Н. Васильев. Интеллектуальное управление динамическими системами / С. Н. Васильев, А. К. Жерлов, Е. А. Федосов, Б. Е. Федунов - М. Физико-математическая литература, 2000. - 352 с.

\bibitem{lar2013} А. А. Ларионов. Программные технологии для эффективного поиска логического вывода в исчислении позитивно-образованных формул / А. А. Ларионов, Е. А. Черкашин – Иркутск : Изд-во ИГУ, 2013. – 104 c.
\end{thebibliography}

\begin{englishtitle}

\title{Logical computer simulation of a group of elevators}

\author{%
  S.~Arlyapov\inst{1}
  \and
  E.~Cherkashin\inst{2,3,4}
}

\institute{
  Irkutsk state university of railway engineering, Irkutsk, Russia
  \and
  V.~M.~Matrosov's Institute of system dynamics and control theory SB RAS, Irkutsk, Russia
  \and National research Irkutsk state technical university, Irkutsk, Russia
  \and
  Institute of mathematics, economics and informatics of ISU, Irkutsk, Russia\\
  \email{sava\_arl@mail.ru,eugeneai@icc.ru}
}

\maketitle
\begin{abstract}
A problem of development of a computer simulation model for a group of elevators with logical programming is considered.  The model consists of two parts. The first one is a physical model of the elevators, and the second one is a control synthesis.  Test simulations are produced.

\keywords{automatic control system, computer simulation, logical programming, Prolog}
\end{abstract}
\end{englishtitle}
