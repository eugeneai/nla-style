%%%%%%%%%%%%%%%%%%%%%%%%%%%%%%%%%%%%%%%%%%%%%%%%%%%%%%%%%%%%%%%%%%%%%%%%
%
%  This is the template file for the 6th International conference
%  NONLINEAR ANALYSIS AND EXTREMAL PROBLEMS
%  June 25-30, 2018
%  Irkutsk, Russia
%
%%%%%%%%%%%%%%%%%%%%%%%%%%%%%%%%%%%%%%%%%%%%%%%%%%%%%%%%%%%%%%%%%%%%%%%%

%  Верстка статьи осуществляется на основе стандартного класса llncs
%  (Lecture Notes in Computer Sciences), который корректируется стилевым
%  файлом конференции.
%
%  Скомпилировать файл в PDF можно двумя способами:
%  1. Использовать pdfLaTeX (pdflatex), (LaTeX+DVIPS не работает);
%  2. Использовать LuaLaTeX (XeLaTeX будет работать тоже).
%  При использовании LuaLaTeX потребуются TTF- или OTF-шрифты CMU
%  (Computer Modern Unicode). Шрифты устанавливаются либо пакетом
%  дистрибутива LaTeX cm-unicode
%              (https://www.ctan.org/tex-archive/fonts/cm-unicode),
%  либо загрузкой и установкой в операционной системе, адрес страницы:
%              http://canopus.iacp.dvo.ru/%7Epanov/cm-unicode/
%  Второй вариант не будет работать в XeLaTeX.
%
%  В MiKTeX (дистрибутив LaTeX для ОС Windows):
%  1. Пакет cm-unicode устанавливается вручную в программе MiKTeX Console.
%  2. Для верстки данного примера, а именно, картинки-заглушки необходимо,
%     также вручную, загрузить пакет pgf. Этот пакет используется популярным
%     пакетом tikz.
%  3. Тест показал, что остальные пакеты MiKTeX грузит автоматически (если
%     ему разрешено автоматически грузить пакеты). Режим автозагрузки
%     настраивается в разделе Settings в MiKTeX Console.
%
%
%  Самый простой способ сверстать статью - использовать pdfLaTeX, но
%  окончательный вариант верстки сборника будет собран в LuaLaTeX,
%  так как результат получится лучшего качества, благодаря пакету microtype и
%  использованию векторных шрифтов OTF вместо растровых pdfLaTeX.
%
%  В случае возникновения вопросов и проблем с версткой статьи,
%  пишите письма на электронную почту: eugeneai@irnko.net, Черкашин Евгений.
%
%  Новые варианты корректирующего стиля будут доступны на сайте:
%        https://github.com/eugeneai/nla-style
%        файл - nla.sty
%
%  Дальнейшие инструкции - в тексте данного шаблона. Он одновременно
%  является примером статьи.
%
%  Формат LaTeX2e!

\documentclass[12pt]{llncs}  % Необходимо использовать шрифт 12 пунктов.

% При использовании pdfLaTeX добавляется стандартный набор русификации babel.
% Если верстка производится в LuaLaTeX, то следующие три строки надо
% закомментировать, русификация будет произведена в корректирующем стиле автоматом.
\usepackage[T2A]{fontenc}
\usepackage[utf8]{inputenc} % Кодировка utf-8, win1251 (cp1251) не тестировалась.
\usepackage[english,russian]{babel}

% Для верстки в LuaLaTeX текст готовится строго в utf-8!

% В операционной системе Windows для редактирования в кодировки utf-8
% можно использовать программы notepad++ https://notepad-plus-plus.org/,
% techniccenter http://www.texniccenter.org/,
% SciTE (самая маленькая по объему программа) http://www.scintilla.org/SciTEDownload.html
% Подойдет также и встроенный в свежий дистрибутив MiKTeX редактор
% TeXworks.

% Добавляется корректирующий стилевой файл строго после babel, если он был включен.
% В параметре необходимо указать russian, что установит не совсем стандартные названия
% разделов текста, настроит переносы для русского языка как основного и т.п.

\usepackage{todonotes} % Этот пакет нужен для верстки данного шаблона, его
                       % надо убрать из вашей статьи.

\usepackage[russian]{nla}

% Многие популярные пакеты (amsXXX, graphicx и т.д.) уже импортированы в корректирующий стиль.
% Если возникнут конфликты с вашими пакетами, попробуйте их отключить и сверстать
% текст как есть.
%
%


% Было б удобно при верстке сборника, чтобы названия рисунков разных авторов не пересекались.
% Чтоб минимизировать такое пересечение, рисунки можно поместить в отдельную подпапку с
% названием - фамилией автора или названием статьи.
%
% \graphicspath{{ivanov-petrov-pics/}} % Указание папки с изображениями в форматах png, pdf.
% или
% \graphicspath{{great-problem-solving-paper-pics/}}.


\begin{document}

% Текст оформляется в соответствии с классом article, используя дополнения
% AMS.
%

\title{Название доклада\thanks{Работа выполнена при поддержке РФФИ (РНФ, другие фонды), проект \textnumero~00-00-00000.}}
% Первый автор
\author{И.~О.~Фамилия\inst{1}  % \inst ставит циферку над автором.
  \and  % разделяет авторов, в тексте выглядит как запятая.
% Второй автор
  И.~О.~Фамилия\inst{2}
  \and
% Третий автор
  И.~О.~Фамилия\inst{1}
} % обязательное поле

% Аффилиации пишутся в следующей форме, соединяя каждый институт при помощи \and.
\institute{Институт (название в краткой форме), Город, Страна\\
  \email{email}
  \and   % Разделяет институты и присваивает им номера по порядку.
Институт (название в краткой форме), Город, Страна\\
\email{email}
% \and Другие авторы...
}

\maketitle

\begin{abstract}
Аннотация на русском языке.

\keywords{ключевое слово, другое ключевое слово, \ldots} % в конце списка точка не ставится
\end{abstract}

\section{Основные результаты} % не обязательное поле

Текст доклада на русском языке.

% Рисунки и таблицы оформляются по стандарту класса article. Например,

\begin{figure}[htb]
  \centering
  % Поддерживаются два формата:
  %\includegraphics[width=0.7\linewidth]{figure.pdf} % Растровый формат
  %\includegraphics[width=0.7\linewidth]{figure.png} % Векторный и растровый формат
  %
  % Векторные рисунки можно рисовать в редакторе Inkscape
  % https://inkscape.org/ru/download/
  % Основной формат этого редактора - SVG, поэтому рисунки необходимо экспортировать в
  % PDF или PNG (с разрешением - минимум 150 dpi, максимум - 300dpi).
  \begin{center}
    \missingfigure[figwidth=0.7\linewidth]{Уберите меня из статьи!}
  \end{center}
  \caption{Заголовок рисунка}\label{fig:example}
\end{figure}

% Современные издательства требуют использовать кавычки-елочки << >>.

% В конце текста можно выразить благодарности, если этого не было
% сделано в ссылке с заголовка статьи, например,
Работа выполнена при поддержке РФФИ (РНФ, другие фонды), проект \textnumero~00-00-00000.
%

% Список литературы оформляется подобно ГОСТ-2008.
% Примеры оформления находятся по этому адресу -
%     https://narfu.ru/agtu/www.agtu.ru/fad08f5ab5ca9486942a52596ba6582elit.html
%

\begin{thebibliography}{9} % или {99}, если ссылок больше десяти.
\bibitem{Gantmakher} Гантмахер~Ф.Р. Теория матриц. М.:~Наука,~1966.

\bibitem{Kholl} Современные численные методы решения обыкновенных дифференциальных уравнений~/ Под~ред.~Дж.~Холл, Дж.~Уатт. М.:~Мир,~1979.

\bibitem{Aleksandrov1} Александров~А.Ю. Об устойчивости сложных систем в критических случаях~// Автоматика и телемеханика. 2001. \textnumero~9. С.~3--13.

\bibitem{Moreau1977} Moreau~J.-J. Evolution problem associated with a moving convex set in a Hilbert space~// J.~Differential~Eq. 1977. Vol.~26. Pp.~347--374.

\bibitem{Semenov} Семенов~А.А. Замечание о вычислительной сложности известных предположительно односторонних функций~// Тр.~XII Байкальской междунар. конф. <<Методы оптимизации и их приложения>>. Иркутск, 2001. С.~142--146.

\end{thebibliography}

% После библиографического списка в русскоязычных статьях необходимо оформить
% англоязычный заголовок.

\begin{englishtitle} % Настраивает LaTeX на использование английского языка
% Этот титульный лист верстается аналогично.
\title{Title of the Paper}
% First author
\author{Name FamilyName1\inst{1}
  \and
  Name FamilyName2\inst{2}
  \and
  Name FamilyName3\inst{1}
}
\institute{Affiliation, City, Country\\
  \email{email}
  \and
Affiliation, City, Country\\
\email{email}}
% etc

\maketitle

\begin{abstract}
Insert your english abstract here. Include 3-6 keywords below.

\keywords{keyword, another keyword, \ldots{}} % в конце списка точка не ставится
\end{abstract}
\end{englishtitle}


\end{document}

%%% Local Variables:
%%% mode: latex
%%% TeX-master: t
%%% End:
