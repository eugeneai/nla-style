%%%%%%%%%%%%%%%%%%%%%%%%%%%%%%%%%%%%%%%%%%%%%%%%%%%%%%%%%%%%%%%%%%%%%%%%
%
%  This is the template file for the 6th International conference
%  NONLINEAR ANALYSIS AND EXTREMAL PROBLEMS
%  June 25-30, 2018
%  Irkutsk, Russia
%
%%%%%%%%%%%%%%%%%%%%%%%%%%%%%%%%%%%%%%%%%%%%%%%%%%%%%%%%%%%%%%%%%%%%%%%%

%  Верстка статьи осуществляется на основе стандартного класса llncs
%  (Lecture Notes in Computer Sciences) , который корректируется стилевым
%  файлом конференции.

\documentclass[12pt]{llncs}

% При использовании
\usepackage[T2A]{fontenc}
\usepackage[utf8]{inputenc}
\usepackage[english,russian]{babel}

% Стилевой файл сборника тезисов конференции основан на классе
%

\usepackage[russian]{nla}


% Многие популярные пакеты уже импортированы в корректирующий стиль

\begin{document}


\title{Название Доклада}
% Первый автор
\author{И.~О.~Фамилия1}
\affiliation{Институт (название в краткой форме), Город, Страна}
\email{email}
% Второй автор
\author{И.~О.~Фамилия2} % обязательное поле
\affiliation{Институт (название в краткой форме), Город, Страна}
\email{email}
% Другие авторы...

\maketitle

\begin{abstract}
Аннотация на русском языке.

{\bf Ключевые слова:} ключевое слово1, другие.
\end{abstract}

\bigskip

\section{Основные результаты} % не обязательное поле

Текст доклада на русском языке.

Работа выполнена при поддержке РФФИ (РНФ, другие фонды), проект \textnumero~00-00-00000.


% Список литературы.
\begin{thebibliography}{9}
\bibitem{1}
% Format for Journal Reference
Author1 N., Author2 N.  {\it Article title}. Journal. Year. Vol.~Volume, No~Number. Pp.~Page numbers.
% Format for books
\bibitem{2}
Author N. Book title. Place: Publisher, year.
% Format for Russian Journal Reference
\bibitem{3} Фамилия И.О. {\it Название статьи}. Журнал. Год. Т.~том,  \textnumero~номер. С.~страницы.
% etc
\end{thebibliography}

\bigskip


% Обязательная информация на английском языке:

\title{Title of the Paper}
% First author
\author{FirstAuthor}
\affiliation{Affiliation, City, Country}
\email{email}
% Second author
\author{SecondAuthor}
\affiliation{Affiliation, City, Country}
\email{email}
% etc

\maketitle

\begin{abstract}
Insert your english abstract here. Include 3-6 keywords below.

{\bf Keywords:} keyword1,  etc.
\end{abstract}


\end{document}

%%% Local Variables:
%%% mode: latex
%%% TeX-master: t
%%% End:
