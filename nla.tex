\documentclass[12pt]{llncs}
\usepackage[T2A]{fontenc}
\usepackage[utf8]{inputenc}
\usepackage[english,russian]{babel}
\usepackage{nla}

%\usepackage[english,russian]{nla}

% \graphicspath{{pics/}} %Set the subfolder with figures (png, pdf).

\usepackage{showframe}
\begin{document}
\babelensure{russian}

\title{Название Доклада}
% Первый автор
\author{И.~О.~Фамилия\inst{1}
  \and
% Второй автор
И.~О.~Фамилия\inst{2}} % обязательное поле
\institute{Институт (название в краткой форме), Город, Страна\\
  \email{email}
  \and
Институт (название в краткой форме), Город, Страна\\
\email{email}}
% Другие авторы...

\maketitle

\begin{abstract}
Аннотация на русском языке.

{\bf Ключевые слова:} ключевое слово1, другие.
\end{abstract}

\section{Основные результаты} % не обязательное поле

Текст доклада на русском языке.

Работа выполнена при поддержке РФФИ (РНФ, другие фонды), проект \textnumero~00-00-00000.


% Список литературы.
\begin{thebibliography}{9}
\bibitem{1}
% Format for Journal Reference
Author1 N., Author2 N.  {\it Article title}. Journal. Year. Vol.~Volume, No~Number. Pp.~Page numbers.
% Format for books
\bibitem{2}
Author N. Book title. Place: Publisher, year.
% Format for Russian Journal Reference
\bibitem{3} Фамилия И.О. {\it Название статьи}. Журнал. Год. Т.~том,  \textnumero~номер. С.~страницы.
% etc
\end{thebibliography}


\end{document}

%%% Local Variables:
%%% mode: latex
%%% TeX-master: nla.tex
%%% End:
