%%%%%%%%%%%%%%%%%%%%%%%%%%%%%%%%%%%%%%%%%%%%%%%%%%%%%%%%%%%%%%%%%%%%%%%%
%
% This is the template file for the 6th International conference
% NONLINEAR ANALYSIS AND EXTREMAL PROBLEMS
% June 25-30, 2018
% Irkutsk, Russia
%
%%%%%%%%%%%%%%%%%%%%%%%%%%%%%%%%%%%%%%%%%%%%%%%%%%%%%%%%%%%%%%%%%%%%%%%%
% The preparation of the article is based on the standard llncs class
% (Lecture Notes in Computer Sciences), which is adjusted with style
% file of the conference.
%
% There are two ways of compilation of the file into PDF
% 1. Use pdfLaTeX (pdflatex), (LaTeX+DVIPS will not work);
% 2. Use LuaLaTeX (XeLaTeX will work too).
% When using LuaLaTeX You will need TTF or OTF CMU fonts
% (Computer Modern Unicode). The fonts are installed with 'cm-unicode' package in
% a distribution of LaTeX % (https://www.ctan.org/tex-archive/fonts/cm-unicode),
% either by downloading and installing these fonts system wide, the address of their page is
% http://canopus.iacp.dvo.ru/%7Epanov/cm-unicode/
% The second option won't work in XeLaTeX.
%
% For MiKTeX (LaTeX distribution for Windows),
%  1. Package 'cm-unicode' is installed manually with the MiKTeX administration Console.
%  2. For the compilation of this example, namely, the stub figure, one will also need to
% download package 'pgf' manually. This package uses in the popular
% package tikz.
%  3. Tests showed that the rest of the required packages MiKTeX loads automatically (if
%     it is allowed). The 'auto download' option is
%     configured in 'Settings' section in MiKTeX Console.
%
%
% The easiest way to compile an article is to use pdfLaTeX, but
% the final layout of the book will be compiled with LuaLaTeX,
% as a result will be of better quality thanks to the package 'microtype' and
% use vector OTF instead of standard raster fonts of pdfLaTeX.
%
% In the case of questions and problems with the article compilation,
% write letters to e-mail: eugeneai@irnok.net, Cherkashin Evgeny.
%
% New version of the correcting style file will be available at the website:
%     https://github.com/eugeneai/nla-style
%     file - nla.sty
%
% Further instructions are in the text body of the template. The template itself
% is an article example.
%
% The LaTeX2e format is used!

% 12 points font size is used.
\documentclass[12pt]{llncs}

% The correcting style file is added.
\usepackage{todonotes}

\usepackage{nla} % This package is needed for compiling
                 % this template, it should be removed
                 % from your article.

% Many popular packages (amsXXX, graphicx, etc.) are already imported in the style file.
% If there is a conflict with your packages, try disabling them and compile
% the text.
%
% It would be convenient in the layout of the proceedings if the file names
% of the figures of different authors do not clash.
% To minimize the clash, the drawings can be placed in a separate subfolder
% named after the author or the title of the paper.
%
% \graphicspath{{ivanov-petrov-pics/}} % specifies the folder with images in png, pdf formats.
% or
% \graphicspath{{great-problem-solving-paper-pics/}}.

\begin{document}

% Text should be formatted in accordance with the 'article' class, using extensions like
% AMS.
%
\title{Title of the Paper\thanks{The research is supported by RFBR (RNF, other funds), project No.~00-00-00000.}}
% First author
\author{Name FamilyName1\inst{1}
  \and
  Name FamilyName2\inst{2}
  \and
  Name FamilyName3\inst{1}
}
\institute{Affiliation, City, Country\\
  \email{email}
  \and
Affiliation, City, Country\\
\email{email}}
% etc

\maketitle

\begin{abstract}
Insert your english abstract here. Include 3-6 keywords below.

\keywords{keyword, another keyword, \emph{etc.}}
\end{abstract}

% at the end of the list, there should be no final dot
\section{The main results}


The text of the report.

% The figures and tables are drawn according to the standard class 'article'.
\begin{figure}[htb]
  \centering

% Two picture formats are supported:
%\includegraphics[width=0.7\linewidth]{figure.pdf} % Raster format
%\includegraphics[width=0.7\linewidth]{figure.png} % Vector and raster format
%
% Vector drawings can be drawn in Inkscape editor
% https://inkscape.org/ru/download/
% The usual format of the editor is SVG, so the drawings must be exported in
% PDF or PNG (with a resolution of minimum 150 dpi, and maximum of 300 dpi).
  \begin{center}
    \missingfigure[figwidth=0.7\linewidth]{Remove me from the article!} \end{center}
  \caption{Caption of the figure}\label{fig:example}
\end{figure}

% At the end of the text, acknowledgments are expressed, if you haven't
% made a footnote from the title. For example, we can write
The research is carried on with support of RFBR (RNF, other funds), project No.~00-00-00000.

\begin{thebibliography}{9} % or {99}, if there is more than ten references.
\bibitem{DLions1976} Duvaut D., Lions J.L. Inequalities in Mechanics and Phisics. Springer, Berlin, 1976.

\bibitem{Gur1997}  Gurman V.I. The Extension Principle in Optimal Control Problems. 2nd~ed. Fizmatlit, Moscow, 1997.~[In Russian]

\bibitem{Moreau1977} Moreau J.-J. Evolution problem associated with a moving convex set in a Hilbert space. J. Differential Eq.~1977. Vol.~26. Pp.~347--374.

\bibitem{BrKr2013}  Brokate M., Krej\u{c}\'{\i} P. Optimal control of ODE systems involving a rate independent variational inequality. Disc. Cont. Dyn. Syst. Ser.~B. 2013. Vol.~18, no~2. Pp.~331--348.

\bibitem{Karpinski2014} Kapinski J., Deshmukh J, Sankaranarayanan S., Arechiga N. Simulation-guided Lyapunov analysis for hybrid dynamical systems. In Proceedings of the 17th International Conference on Hybrid Systems: Computation and Control (HSCC 2014), Berlin, Germany, 2014. Pp.~133--142.

\bibitem{Forsman1991} Forsman K. Construction of Lyapunov functions using Grobner bases. In Proceedings of the 30th IEEE Conference on Decision and Control, Brighton, UK, 1991. Vol.~1. Pp.~798--799.

\end{thebibliography}
\end{document}

%%% Local Variables:
%%% mode: latex
%%% TeX-master: t
%%% End:
